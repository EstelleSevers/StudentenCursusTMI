\documentclass[12pt,a4paper]{article}
\usepackage[utf8]{inputenc}
\usepackage[T1]{fontenc}
\usepackage[dutch]{babel}
\usepackage{amsmath}
\usepackage{amsfonts}
\usepackage{amssymb}
\usepackage{graphicx}
\usepackage[framemethod=TikZ]{mdframed}

\newenvironment{sidenote}
{
	\begin{center}
		\begin{mdframed}[backgroundcolor=BurntOrange, roundcorner=10pt, linewidth= 0pt]
		}
		{
		\end{mdframed}
	\end{center}
}

\author{Estelle Severs, Maarten Magits}
\title{Studentencursus TMI}
\begin{document}
	\maketitle
	\section{Introductie}
	Toepassingen van de informatica (TMI) wordt ook wel Computationele geometrie (CG) genoemd.
	In het algemeen handelt dit domein over het berekenen en manipuleren van geometrische entiteiten/vormen en een oplossing te vinden voor geometrische problemen.
	Er vallen twee subdomeinen te onderscheiden:
	\begin{enumerate}
		\item \textbf{Combinatorial CG}: De studie van concrete objecten. Behandelt de discrete ruimte.
		\item \textbf{Numerical CG}: De studie van curves en functies. Eerder abstract.
	\end{enumerate}
	Dit OPO gaat grotendeels over de Combinatorial CG met slects een klein deeltje Numerical CG.
	
	\section{Les 1: Convex Omhullende}
	\begin{sidenote}
		Ontwikkelen van een algoritme
	\end{sidenote}
	\subsection{Brute Force algoritme}
	\subsection{Incrementeel algoritme 	(Andrew's algorithm)}
	\subsection{Graham's scan}
	\subsection{Jarvis's March}
	\subsection{Verdeel-en-heers}
	\section{Les 2: Intersecties van lijnstukken}
	\subsection{Brute Force algoritme}
	\subsection{Doorlooplijn algoritme}
	\subsection{Doubly-connected edge 	list}
	\section{Les 3: }
	\section{Les 4: }
	\section{Les 5: }
	\section{Les 6: }
	\section{Les 7: }
	\section{Les 8: }
	\section{Les 9: }
	\section{Les 10: }
	\section{Les 11: }
\end{document}