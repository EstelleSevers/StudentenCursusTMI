\documentclass[12pt,a4paper]{article}
\usepackage[utf8]{inputenc}
\usepackage[T1]{fontenc}
\usepackage[dutch]{babel}
\usepackage{amsmath}
\usepackage{amsfonts}
\usepackage{amssymb}
\usepackage{graphicx}
\usepackage{algpseudocode}
\usepackage{algorithm}
\usepackage{fullpage}
\usepackage{subfigure}
\usepackage{hyperref}


\author{Matthias Kovacic}
\title{TMI - Cheat Sheet Complexiteiten}
\begin{document}
	\maketitle
	\section{Les 1 - Convex Omhullende}
		In de eerste les zijn de volgende algoritmes besproken om de convex omhullende te berekenen:
		\begin{enumerate}
			\item Brute Force
			\item (Andrew's) Incremental Algorithm
			\item Graham Scan
			\item Jarvis March
			\item Divide-\&-Conquer
		\end{enumerate}
	
		\subsection{Brute Force}
			\begin{enumerate}
				\item Tijdscomplexiteit: $O(n^3)$ - Voor de $n^2$ paren moeten er nog $n - 1$ punten gecontroleerd worden.
			\end{enumerate}
		
		\subsection{Incremental Algorithm}
			\begin{enumerate}
				\item Eerst worden alle punten gesorteerd volgens x-coordinaat. Dit vraagt $O(n log (n))$ tijd.
				\item Daarna worden de punten $n$ keer overlopen, wat $O(n)$ tijd vraagt.
				\item Tijdscomplexiteit: $O(n log(n)) + O(n) = O(n log (n))$
			\end{enumerate}
		
		\subsection{Graham Scan}
			\begin{enumerate}
				\item Zoek het punt met laagste y-coordinaat. Dit vraagt $O(n)$ tijd.
				\item Sorteer alle punten volgens poolhoek t.o.v. het punt gevonden in stap 1. Dit vraagt $O(n log (n))$ tijd.
				\item Daarna wordt elk punt overlopen (a la Andrews Style). Dit vraagt $O(n)$.
				\item Tijdscomplexiteit: $O(n) + O(n log(n)) + O(n) = O(n log (n))$
			\end{enumerate}
		
		\subsection{Jarvis March}
			\begin{enumerate}
				\item Zoek het punt met laagste y-coordinaat. Dit vraagt $O(n)$ tijd.
				\item Overloop $n$ punten en bereken hun poolhoek t.o.v. het eerste punt. Zoek het punt met kleinste poolhoek. Dit vraagt opnieuw $O(n)$ tijd. Beschouw het gevonden punt nu als referentiepunt.
				\item Doe dit tot je terug in het oorspronkelijke punt komt met laagste y-coordinaat. Dit vraagt $O(k)$ tijd waarbij $k$ het aantal punten op de convex omhullende is. Elke iteratie worden er dus $n$ punten gecontroleerd.
				\item Tijdscomplexiteit: $O(n) + O(kn) = O(kn)$
				\item Opmerking 1. De tijdscomplexiteit is hier \emph{output-afhankelijk}!
				\item Opmerking 2. $k$ is hier geen constante. De $O(kn)$ term mag dus niet zomaar vereenvoudigd worden naar $O(n)$!
			\end{enumerate}
		
		\subsection{Divide-\&-Conquer}
			\begin{enumerate}
				\item Verdeel de punten in 2 helften, dit vraagt $O(n)$ tijd. 
				\item Bepaal de convex omhullende voor de twee helften. Dit kan met verdere opdeling of een algoritme dat eerder gezien is. Dit vraagt ongeveer $O(n log (n))$ tijd.
				\item De \emph{merge}-stap (e.g. het zoeken van onder/bovenbruggen) kan gebeuren in $O(n)$ tijd. 
				\item Tijdscomplexiteit: $O(n) + O(n log (n)) + O(n) = O(n log (n))$
			\end{enumerate}
	\section{Les 2 - Intersecties van Lijnstukken/DCEL}
		In de tweede les zijn de volgende algoritmes besproken:
		\begin{enumerate}
			\item Brute Force Line Intersection
			\item Bentley-Ottman Algorithm
			\item DCEL Overlay
		\end{enumerate}
	
		\subsection{Brute Force Line Intersection}
			\begin{enumerate}
				\item Tijdscomplexiteit: $O(n^2)$, alle paren van lijnstukken moeten worden gecontroleerd op intersectie.
			\end{enumerate}
		
		\subsection{Bentley-Ottman Algorithm}
			\begin{enumerate}
				\item Opmerking 1. Dit is een sweep-line algoritme. Er moet een event-queue en status bijgehouden worden. We veronderstellen deze om BST (Binary Search Trees) te zijn, wat altijd $O(log (n))$ operaties garandeert.
				\item Opmerking 2. De events zijn de start/eindpunten van de lijnsegmenten. Bij $n$ segmenten zijn er dus $n$ start-events en $n$ eind-events.
				\item Een start-event kan worden behandelt in $O(log (n))$ tijd. Het komt $n$ keer voor, dus alle start-events behandelen vraagt $O(n log (n))$ tijd.
				\item Eind-events zijn analoog aan start-events: $O(n log (n))$. 
				\item Het andere soort events zijn de intersectie-events. Zo zijn er $k$ (ook weer onbekend, Cfr. Jarvis March). Alle intersectie-events behandelen vraagt dus $O(k log n)$ tijd.
				\item Tijdscomplexiteit: $O(n log (n)) + O(n log (n)) + O(k log (n)) = O((n + k) log (n))$ 
				\item Opmerking 3. De tijdscomplexiteit is hier \emph{output-afhankelijk}!
			\end{enumerate}
		
		\subsection{DCEL Overlay}
			\begin{enumerate}
				\item Opmerking 1. Het algoritme zelf staat volledig uitgeschreven in het boek. Het maakt gebruik van een sweepline en van supergrafen. 
				\item Opmerking 2. De complexiteit van een subdivisie (e.g. DCEL) is gegeven door: $c = v + e + f$ waarbij $v$ het aantal vertices, $e$ het aantal edges en $f$ het aantal faces is. 
				\item Tijdscomplexiteit: $O(n log (n) + k log (n))$ waarbij $n = n_1 + n_2$ met $n_1$ de complexiteit van de eerste subdivisie en $n_2$ die van de tweede. $k$ is de complexiteit van de geconstrueerde overlay.
				\item Opmerking 3. De tijdscomplexiteit is hier \emph{output-afhankelijk}!
			\end{enumerate}     
	\section{Les 3 - Veelhoek-triangulatie}    
		In de derde les zijn de volgende algoritmes besproken:
		\begin{enumerate}
			\item Y-monotone polygon
			\item Polygon triangulation
		\end{enumerate}                
	
		\subsection{Y-monotone polygon}
			\begin{enumerate}
				\item Opmerking 1. Dit is opnieuw een sweepline-algoritme. Er wordt nergens het "bewijs" van complexiteit geleverd.
				\item Tijdscomplexiteit: $O(n log (n))$ met $n$ het aantal vertices in de polygon. 
			\end{enumerate}
		
		\subsection{Polygon Triangulation}
			\begin{enumerate}
				\item Alle vertices overlopen kost $O(n)$ tijd. Het kan bewezen worden (zie boek) dat ook de verbindingen maken maximaal lineaire tijd kost.
				\item Tijdscomplexiteit: $O(n)$
			\end{enumerate}
		
	\section{Les 4 - Half Plane Intersection}
		In de vierde les zijn de volgende algoritmes besproken:
		\begin{enumerate}
			\item Common intersection of a set of $n$ half-planes
			\item Incremental intersection of a set of $n$ half-planes
			\item Kleinst omsluitende cirkel
		\end{enumerate}    
	
		\subsection{Common Intersection (Sweepline) of Halfplanes}
			\begin{enumerate}
				\item Opmerking 1. Elk halfvlak is gedefinïeerd als een verzameling van punten die voldoen aan een constraint van de vorm: $a_ix + b_iy \leq c_i$.
				\item Opmerking 2. Een constraint is een lineaire vergelijking, vandaar de naam "lineair programmeren"
				\item Opmerking 3. We kunnen dit ook recursief berekenen, maar dan komen we uit op een algoritme dat $O(n log^2 (n))$ tijd vraagt.
				\item We splitsen de set van halfvlakken in twee en gaan recursief te werk (Cfr. Merge Sort). Intuïtief is aan te voelen dat dit een logaritmische $O(log (n))$ complexiteit zal geven. We bekijken enkel nog de tijd nodig om de merge-stap uit te voeren. 
				\item We gebruiken een sweepline algoritme dat de linker/rechterkant van elke set van halfvlakken bijhoudt (We berekenen de intersectie van twee sets van halfvlakken, niet van de halfvlakken zelf!).
				\item Er is $O(n)$ tijd nodig om te intersectie van twee convexe polygons te vinden.
				\item Tijdscomplexiteit: $O(n) \times O(log (n)) = O(n log (n))$.
			\end{enumerate}
	
		\subsection{Incremental Intersection of Halfplanes}
			\begin{enumerate}
				\item Opmerking 1. We hebben in het vorige algoritme gezocht naar een regio waar alle halfvlakken geldige punten hebben. We kunnen de logaritmische factor wegwerken door de oplossing te maximaliseren.
				\item Opmerking 2. We starten met twee halfvlakken $m_1$ en $m_2$. We zoeken hun snijpunt $v_0$ (als ze er één hebben), dit is nu de "optimale vertex". We permuteren nu de andere halfvlakken en voegen deze toe. Als de optimale vertex tot een nieuw halfvlak behoort doen we niets, anders zoeken we naar een nieuwe optimale vertex.
				\item We kunnen een optimale vertex vinden in de $i$-de stap met $O(i)$ tijd. 
				\item Opmerking 3. Als we de permutatiestap niet doen, krijgen we worst case een $\sum_{i=1}^{n}O(i) = O(n^2)$ algoritme.
				\item We bekijken de kans dat na de permutatie de $i$-de stap een dure stap is (e.g. we zoeken de optimale vertex). Aangezien er 2 mogelijkheden zijn die deze stap induceren, is die kans $\frac{2}{i}$. De kans dat de stap dan goedkoop is is $\frac{i - 2}{i}$. 
				\item Tijdscomplexiteit wordt dan: $\sum_{i=1}^{n}O(i)\frac{i}{2} = O(n)$. 
				\item Opmerking 4. Dit is de \emph{expected} running time. In werkelijkheid kunnen we nog altijd $O(n^2)$ tijd nodig hebben als de permutatie slecht is.
			\end{enumerate}
			
		\subsection{Kleinst Omsluitende Cirkel}
			\begin{enumerate}
				\item We beginnen met 3 punten. De cirkel $D_n$ is dan uniek. Als we een punt toevoegen in de $i + 1$-stap, kan dit punt reeds binnen $D_i$ liggen, we doen niets. Als het punt buiten de cirkel ligt, dan heeft de nieuwe cirkel $D_{i+1}$ dit punt op de rand. We hebben nu een nieuw probleem: de kleinst omsluitende cirkel met 1 punt op de rand. We gebruiken dezelfde strategie. Het probleem duikt opnieuw op voor een kleinst omsluitende cirkel met 2 punten op de rand. Dit laatste kost $O(n)$ tijd.
				\item Hoeveel tijd vraagt het om de kleinst omsluitende cirkel te berekenen met 1 punt op de rand? Er zijn twee punten die op de rand liggen dus de kans dat een punt tot de rand behoort is $\frac{2}{i}$ (Cfr. incrementele constructie halfvlakken). Dit vraagt $O(n)$ tijd. 
				\item Voor de kleinst omsluitende cirkel gebruiken we dezelfde techniek, wat dus ook $O(n)$ tijd vraagt. 
				\item Tijdscomplexiteit: $O(n)$
			\end{enumerate}
	\section{Les 5 - KD-Bomen \& Range Trees}
		In de vijfde les zijn de volgende algoritmes besproken:
		\begin{enumerate}
			\item 1D Interval Query
			\item Constructie KD-boom
			\item k-D Interval Query in KD-boom
			\item Constructie Range Tree
			\item Query in Range Tree
		\end{enumerate}   
	
		\subsection{1D Interval Query}
			\begin{enumerate}
				\item Om data bij te houden gebruiken we een BST. Dit garandeert dat alle operaties gebeuren in $O(log (n))$ tijd.
				\item Opmerking 1. Het zou ondertussen algemene kennis moeten zijn dat een BST $O(n)$ geheugen vraagt en $O(n log (n))$ tijd om op te bouwen.
				\item Als we een interval van data willen opvragen, zoeken we naar $x_{min} en x_{max}$ in de BST in logaritmische tijd. 
				\item Onderweg tijdens deze zoektocht moeten we echter enkele subbomen rapporteren. Deze overlopen vraagt lineaire tijd in het aantal data-punten: $O(k)$. 
				\item Tijdscomplexiteit: $O(log (n)) + O(k) = O(log (n) + k)$
			\end{enumerate}
		
		\subsection{Constructie KD-boom}
		\begin{enumerate}
			\item Analoog aan het zoeken van de common intersection van halfplanes en merge-sort gebruiken we een recursief algoritme. We houden splitsingslijnen bij in een BST dus alle operaties hierop zijn opnieuw gegarandeert te gebeuren in $O(log (n))$ tijd. 
			\item Elke stap zoeken we de mediaan. Dit kan in constante tijd gebeuren indien we de data sorteren. Dit vraagt $O(n log (n))$ tijd. Om geen kwadratisch logaritme te krijgen gebeurt dit sorteren in een pre-process. 
			\item Tijdscomplexiteit: $O(n log (n)) + O(n log (n)) = O(n log (n))$
			\item Opmerking 1. Het geheugen is opnieuw $O(n)$ voor een BST. 
		\end{enumerate}
	
		\subsection{k-D Interval Query in kD-boom}
		\begin{enumerate}
			\item Opmerking 1. Elke knoop in de BST definïeert een regio de k-D representatie. We kunnen dus opnieuw een recursief zoekalgoritme gebruiken. 
			\item We weten van de 1D Interval Query dat het rapporteren van subbomen lineair is in het aantal datapunten $i$: $O(i)$. 
			\item We zoeken nu hoeveel knopen we bezoeken die slechts partieel in het zoekinterval liggen. We onderzoeken knopen verder indien de regio van een kinderknoop van de huidige knoop het interval intersecteert. 
			\item Dit komt neer op het zoeken naar het aantal intersectie tussen een willekeurige horizontale/verticale lijn met de kD-boom. Het aantal regio's zal verdubbelen elke $k$ niveau's in de boom. Alle doorsnede van de regio's vormen dus een boom met hoogte $\frac{1}{k}log (n)$. Het totaal aantal regio's is dus begrensd door $2^{\frac{1}{k}log (n)}$. Als we dit uitrekenen voor $k = 2$ bekomen we: $O(2^{\frac{1}{2}log (n)}) = O(\sqrt{n})$.
			\item Tijdscomplexiteit 2D-boom: $O(\sqrt{n} + i)$
			\item Tijdscomplexiteit kD-boom: $O(n^{1-\frac{1}{k}} + i)$
		\end{enumerate}
	
		\subsection{Constructie Range Tree}
		\begin{enumerate}
			\item Opmerking 1. Een range tree heeft mogelijks betere tijdscomplexiteit om op te zoeken, maar dit gaat ten koste van het geheugen!
			\item Elk datapunt is maximaal 1 keer opgeslagen per niveau van de boom: $O(n)$. Er zijn $O(log n)$ niveau's waarop het punt wordt opgeslagen.
			\item Ruimtecomplexiteit: $O(n log (n))$ 
		\end{enumerate}
	
		\subsection{Query in Range Tree}
		\begin{enumerate}
			\item Op elke knoop moeten we beslissen of we verder afdalen in de boom of dat we een subboom oproepen. We doen hier de analyse voor 2D en generaliseren later.
			\item We weten als we een subboom oproepen we eigenlijk een 1D Interval Query doen. Dit vraagt $O(log (n) + k)$ tijd, met $k$ het aantal te rapporteren datapunten.
			\item Dit kan mogelijks voor elke knoop: $\sum_{v}O(log (n) + k_v)$, met $v$ een knoop in de huidige boom. 
			\item De som van alle $k_v$ is $k$, het totale aantal gerapporteerde punten. We weten ook dat de lengte van de zoekpaden voor de query in deze boom $O(log (n))$ tijd vragen. De som $\sum_{v}O(log (n)) = O(log^2 (n))$
			\item Tijdscomplexiteit: $O(log^2 (n) + k)$µ
			\item Opmerking 1. In $d$ dimensies wordt dit voor het geheugen: $O(n log^{d-1} (n))$, voor de constructietijd $O(n log^{d-1} (n))$ en voor het zoeken $O(log^{d} (n) + k)$. 
		\end{enumerate}
	
	\section{Les 6 - Puntlocatie}
		In de zesde les zijn de volgende algoritmes besproken:
		\begin{enumerate}
			\item Puntlocatie m.b.v. Slabs
			\item Opstellen/Query in Trapeziummap
		\end{enumerate}
	
		\subsection{Puntlocatie m.b.v. Slabs}
			\begin{enumerate}
				\item Opmerking 1. We gaan ervan uit dat de polygon in general position is. Dit wilt zeggen dat er geen vertices zijn met dezelde x-coordinaat. Indien dit wel het geval is kunnen we een "shear"-operatie uitvoeren op elke vertex van de polygon:
				$$shear(S) = \forall v \in S: v \rightarrow v_s : \begin{bmatrix}
					v_x\\
					v_y
				\end{bmatrix} \mapsto 
				\begin{bmatrix}
					v_x + \epsilon v_y\\
					v_y
				\end{bmatrix}$$
				\item We overlopen nu in $O(n)$ tijd alle vertices en houden 'verticale slabs' bij. Dit zijn een soort van verticale stroken. Hier is $O(2n) = O(n)$ geheugen voor nodig. 
				\item Als we nu een query voor een punt $q$ willen uitvoeren, kijken we naar de x-coordinaat van $q$. We kunnen in $O(log (n))$ tijd bepalen in welke slab $q$ zal liggen (Cfr. Binary Search).
				\item Binnen elke slab bevatten de edges een pointer naar de originele face. Het zal ook $O(log (n))$ tijd vragen om te kijken waar $q$ ligt binnen deze slab. Binnen deze slab is opnieuw $O(n)$ geheugen nodig.
				\item Tijdscomplexiteit voor een query: $O(log (n))$
				\item Ruimtecomplexiteit: $O(n^2)$	
			\end{enumerate}
		
		\subsection{Opstellen/Query in Trapeziummap}
			De uitleg hier gaat alleen over de complexiteit! We beginnen met het opstellen en gaan daarna over op de analyse van een query. Het opstellen bestaat ook uit twee aparte analyses: de ruimte en tijdscomplexiteit. We beginnen met de ruimtecomplexiteit:
			
			\subsubsection{Ruimtecomplexiteit - Opbouw}
				\begin{enumerate}
					\item De trapeziummap wordt bijgehouden m.b.v. een DAG of \emph{directed acyclic graph} $D$. Initieel bevat deze alleen de boundingbox $R$. Op de $i$-de iteratie is $D = D_i$.
					\item We onderzoeken nu dus de complexiteit van een lijnsegment $s_i$ toe te voegen aan $D_{i-1}$.
					\item We zoeken eerst de trapezia in de vorige trapeziummap die snijden met het nieuwe lijnsegment. Dit zijn er $k_i$ en deze overlopen (m.b.v. $right(\Delta)$) gebeurt in lineaire tijd: $O(k_i)$. 
					\item In de tweede stap zullen in $D_{i-1}$ $k_i$ bladeren worden verwijdert en zullen er $k_i$ nieuwe interne knopen en $k_i$ nieuwe bladeren toegevoegd worden. De maximale diepte wijzigt met $3$. 
					\item Hiermee weten we weinig over de complexiteit van het toevoegen van het segment zelf. Aangezien het algoritme een RIC is (\emph{random incremental construction}, Cfr. Les 4), kunnen we opnieuw gaan kijken naar de \emph{expected} tijdscomplexiteit. 
					\item We weten dat elk segment evenveel kans had op als laatste toegevoegd te worden (RIC). We  vragen ons nu af hoeveel extra trapezia een lijnsegment kan toevoegen indien het als laatste werd toegevoegd:
					$$K_i = \sum_{j=1}^{i}[\#\textrm{trapezia aangemaakt als $s_j$ laatst was}]$$
					Hierbij is $K_i$ de som voor het totaal aantal lijnsegmenten toegevoegd.
					\item We weten dat elke trapezium kan gemaakt worden door 4 lijnsegmenten: $top(\Delta)$, $bottom(\Delta)$, $right(\Delta)$, $left(\Delta)$.
					\item Het is ook bewezen dat op iteratie $i$, er maximaal $3i + 1$ trapezia kunnen bestaan na het toevoegen van $i$ lijnsegmenten. We kunnen dus de expressie van daarnet herschrijven naar:
					$$K_i = \sum_{\Delta \in \tau_i}[\#\textrm{segmenten die $\Delta$ zouden aanmaken}] \leq \sum_{\Delta \in \tau_i}4 = 12i + 4$$
					\item Het gemiddeld aantal trapezia aangemaakt is dus $\frac{12i + 4}{i} \approx 13$ (voor grote $i$).
					\item Het verwacht aantal nieuwe eindknopen in de graaf is dus begrensd met 13. Als er $n$ lijnsegmenten zijn bekomen we dus:
					\item \emph{expected} Ruimtecomplexiteit: $O(13n) = O(n)$
				\end{enumerate}
		
			We leiden nu de tijdscomplexiteit om de structuur op te bouwen af.
			
			\subsubsection{Tijdscomplexiteit - Opbouw}
				\begin{enumerate}
					\item Deze afleiding is eigenlijk triviaal. We voegen punten toe aan een soort van boom, dus ergens is intuïtief aan te voelen dat dit een logaritmische tijdscomplexiteit zal hebben:
					\item Tijdscomplexiteit: $O(n log (n))$
				\end{enumerate}
		
			Tot slot bepalen we de tijdscomplexiteit voor een query uit te voeren:
			
			\subsubsection{Tijdscomplexiteit - Query}
				\begin{enumerate}
					\item We gaan er nog altijd van uit dat we de iteraties zelf bekijken!
					\item We kunnen ons afvragen hoeveel van de $i$ lijnsegmenten het zoekpad voor een punt langer maken. Een zoekpad wordt enkel langer als de trapezium waartoe $q$ behoort verwijdert en aangepast werd. Er zijn 4 lijnsegmenten die dit kunnen veroorzaken (zie vorige analyse), dus de kans dat dit gebeurt is $\frac{4}{i}$.
					\item We beschouwen nu opnieuw een som:
					$$\sum_{i=1}^{n}[\textrm{zoekpad wordt langer dankzij $i$-de toevoeding}] \leq \sum_{i=1}^{n} \frac{4}{i}$$
					$$= 4 \sum_{i=1}^{n}\frac{1}{i}$$
					$$\leq 4(1 + log_e(n))$$
					\item \emph{expected} Tijdscomplexiteit: $O(n log (n))$
				\end{enumerate}
			
	\section{Les 7 - Voronoi Diagramma}
		In les 7 werd enkel het algoritme van Fortune nader bekeken. Het algoritme zelf is zeer complex maar de tijdscomplexiteit is dat niet.
		
		\begin{enumerate}
			\item Opmerking 1. We gebruiken het algoritme van Fortune. Dit is een sweepline algoritme. We houden een status bij in de vorm van een BST en een event queue in de vorm van een priority queue.
			\item Als we $n$ punten hebben, weten we dat er $n$ \emph{site}-events zijn. Door het gebruik van een BST zijn alle operaties hierop begrensd door $O(log (n))$. De \emph{site}-events worden dus allemaal afgehandeld in $O(n log (n))$ tijd. 
			\item Het andere type events zijn \emph{circle}-events. Elk \emph{circle}-event definieert een Voronoi-vertex. We kunnen bewijzen dat er maximaal $2n - 5$ Voronoi-vertices zullen zijn in het hele Voronoi-diagramma. Ook hier zijn de operaties op de BST begrensd door $O(log (n))$. Er zijn $O(2n - 5)$ events, die dus allemaal afgehandeld worden in $O(n log (n))$ tijd.
			\item Tijdscomplexiteit: $O(n log (n)) + O(n log (n)) = O(n log (n))$.				
			\item Opmerking 2. Het is aan te tonen dat het sorteren van een een set getallen reduceerbaar is naar het construeren van het Voronoi-diagramma. Het is dus niet mogelijk een algoritme te vinden dat het beter doet dan $O(n log (n))$. 
		\end{enumerate}
	
	\section{Les 8 - Delaunay Triangulatie}
		In les 8 werd de Delaunay triangulatie besproken. De analyse moet niet gekend zijn, enkel de resultaten. 
	
		\begin{enumerate}
			\item \emph{expected} Tijdscomplexiteit: $O(n log (n))$
			\item \emph{expected} Ruimtecomplexiteit: $O(n)$
		\end{enumerate}	
		
	\section{Les 9 - Béziercurven}
		In les 9 is er geen enkele afleiding voor complexiteit gezien. 
		
	\section{Les 10 - Binary Space Partitions}
		In de tiende les zijn de volgende algoritmes besproken:
		\begin{enumerate}
			\item Fragmentatie in BSP
			\item Constructie BSP
			\item Phase 1 Construction Low-Density BSP
		\end{enumerate}
	
		\subsection{Fragmentatie in BSP}
			\begin{enumerate}
				\item We bewijzen dat het \emph{expected} aantal fragmenten gegenereerd door het BSP-algoritme $O(n log (n))$ is. 
				\item We beginnen met een notie van afstand. Als we segment $s_i$ toevoegen kan het zijn dat een segment $s_j$ 'beschermd' tegen deze splitsing. We definiëren nu de afstand:
				$$dist_{s_i}(s_j) = \begin{cases}
					\text{\#segments intersecting $l(s_i)$ tussen $s_i$ en $s_j$ als $l(s_i)$ $s_j$ intersecteerd}\\
					+\infty \text{ in elk ander geval}
				\end{cases}$$
				\item Voor elk eindige afstand zijn er maximaal twee zo'n segmenten met die afstand. Eentje langst beide kanten van $s_i$.
				\item We berekenen nu de kans dat de splitsing $l(s_i)$ $s_j$ zal snijden:
				$$P[l(s_i) \text{ snijdt $s_j$}] \leq \frac{1}{dist_{s_i}(s_j) + 2}$$
				\item Op basis van deze kans berekenen we het aantal snijdingen gegenereerd door $s_i$:
				$$E[\text{\# snijdingen dankzij $s_i$}] \leq \sum_{j \neq i}\frac{1}{dist_{s_i}(s_j) + 2}$$
				$$\leq 2\sum_{k = 0}^{n-2}\frac{1}{k + 2}$$
				$$\leq 2 ln (n)$$
				\item \emph{expected} aantal fragmenten: $O(2n log (n)) = O(n log (n))$
				\item De verwachtte grootte van de boom is $O(n + 2n log (n))$ 
			\end{enumerate}
	
		\subsection{Constructie BSP}
			\begin{enumerate}
				\item We berekenen de \emph{expected} tijdscomplexiteit om de BSP op te stellen. 
				\item Voor elke splitsing overlopen we een aantal segmenten/fragmenten in de huidig geconstrueerde boom. Dit zijn er maximaal $O(n)$ (kan alleen kleiner worden).
				\item Het aantal recursieve oproepen is ook beperkt door het aantal fragmenten $O(n log (n))$. 
				\item Tijdscomplexiteit: $O(n) * O(n log (n)) = O(n^2 log (n))$
			\end{enumerate}
		
		\subsection{Phase 1 Construction Low-Density BSP}
			\begin{enumerate}
				\item We moeten bewijzen dat de eerste fase van het algoritme om een low-density BSP te bekomen $O(\frac{n}{k})$ bladknopen heeft waar elke bladknoop maximaal $k + 4\lambda$ objecten intersecteert.
				
				\item Het aantal bladeren is $1 + $ het aantal interne knopen. We definiëren $N(m)$ = het aantal interne knopen voor een BSP boom met $m$ \emph{guards}. Er zijn 4 quadranten met elks $m_i$ guards:
				$$N(m) = \begin{cases}
					0 \iff m \leq k\\
					3 + \sum_{i \in I}N(m_i) \iff \text{ in elk ander geval}
				\end{cases}$$
				(\emph{dit is echt een louche stap...})
				\item We werken uit:
				$$N(m_i) \leq max(0, \frac{6m}{k} - 3)$$
				
				\item Als nu $m \leq k$, dan geldt de stelling: triviaal.
				\item Als nu $m > k$ moeten we kijken naar welke stap we doen.
				\item $m > k$ \& Quadtree-Split (2 of meer quadranten $> k$ guards):
				$$N(m) \leq 3 + \sum_{i \in I}N(m_i) \leq 3 + (\sum_{i \in I}(\frac{6m_i}{k})) - 3card(I) \leq \frac{6m}{k} - 3$$
				\item $m > k$ \& Krimp-Stap (1 quadrant $< m - k$ en 3 quadranten $< k$): 
				$$N(m) \leq 3 + N(m - k) \leq 3 + (\frac{6(m - k)}{k - 3}) \leq \frac{6m}{k} - 3$$
				\item \emph{Dit is gwn letterlijk copy paste ik begrijp dit niet...}
				\item Wat is nu het aantal mogelijke objecten per bladknoop?
				\item Een bladknoop bevat maximaal $k$ guards. Dit zijn $k + 4\lambda$ objecten indien de bladknoop een vierkant is in de semi-Quadtree. Dit zijn er maximaal $k + 4\lambda$ in de Quadtree als het een rechthoek is.
			\end{enumerate}
		
	\section{Les 11 - Motion Planning \& Visibility Graphs}
		In de laatste les zijn de volgende algoritmes besproken:
		\begin{enumerate}
			\item Constructie Configuratieruimte bij Puntrobots
			\item Constructie Minkowski-veelhoek
			\item Constructie Visibility Graph
		\end{enumerate}
	
		\subsection{Constructie Configuratieruimte bij Puntrobots}
			\begin{enumerate}
				\item We stellen de configuratieruimte voor door een trapziummap. Zoals gezien in les 6 is deze construeerbaar in $O(n log (n))$ tijd. 
				\item We verwijderen de segmenten binnen de objecten. Dit vraagt lineaire tijd $O(n)$.
				\item Tijdscomplexiteit: $O(n log (n)) + O(n) = O(n log (n))$
				\item Opmerking 1. Het is mogelijk een roadmap te construeren in lineaire tijd. Hierna is het ook mogelijk om met Breadth-First-Search een pad te zoeken in deze roadmap. We kunnen dus padplanning doen in $O(n)$ mits we de preprocess stap nemen om de trapeziummap en roadmap op te stellen.
			\end{enumerate}µ
		\subsection{Minkowski-veelhoek}
			\begin{enumerate}
				\item Opmerking 1. De hele afleiding moet niet gekend zijn, enkel intuïtief.
				\item Twee convexe veelhoeken $P$ en $Q$ met $n$ en $m$ vertices respectievelijk. Het is aan te voelen dat we gebruik makend van het algoritme in de les elke vertex van $P$ en elke vertex van $Q$ éénmaal gebruiken. Dit wordt dus $O(n + m)$.
				\item Tijdscomplexiteit: $O(n + m)$
			\end{enumerate}
		\subsection{Constructie Visibility Graph} 
			\begin{enumerate}
				\item Over het algemeen is het opstellen van een visibiliteitsgrafe $O(n^3)$. Dit komt omdat we elk paar van vertices (= $(2n)^2$) gaan testen op intersectie met elke edge van de grafe (= $n$). 
				\item We kunnen het niet veel beter doen. We krijgen hoogstens één $n$ weggewerkt door een roterende sweepline te gebruiken.
				\item De roterende sweepline heeft als events de vertices van de graaf en houdt op de status (een BST op operaties in $O(log (n))$ te garandere) de volgorde van intersectie bij. Op deze manier weten we welke edges/vertices de sweepline ziet gegeven een bepaalde rotatie.
				\item Aangezien alle operaties in de BST $O(log (n))$ zijn en er nog altijd $(2n)^2$ events zijn, zal het opstellen van de visibiliteitsgrafe een tijdscomplexiteit hebben:
				\item Tijdscomplexiteit $O(n^2 log (n))$. 
				\item Opmerking 1. Dit is het best dat we kunnen doen.
			\end{enumerate}
\end{document}